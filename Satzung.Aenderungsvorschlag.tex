\documentclass[11pt,DIV12]{scrartcl}
%
\usepackage[utf8]{inputenc}
\usepackage[T1]{fontenc}
%\usepackage[lf,minionint]{MinionPro}
%\usepackage[toc,eqno,enum,bib]{tabfigures}
\usepackage[ngerman]{babel}
\usepackage{graphicx}
%
\setkomafont{disposition}{\normalcolor\rmfamily}
\renewcommand*{\labelenumi}{(\theenumi)}
\renewcommand*{\theenumii}{\arabic{enumii}.}
\renewcommand*{\labelenumii}{\theenumii}
\addtolength{\leftmargini}{-\parindent}
\addtolength{\leftmargini}{3pt}
%
\typearea[current]{current}
%
\begin{document}
\begin{center}
\Huge
\textbf{Änderungsvorschlag zur Satzung der warpzone e.V.}

\vspace{4mm}
\noindent
\Large
Stand: September 2015
\end{center}

\vspace{2mm}
\normalsize
\section*{§~1 Name, Sitz, Geschäftsjahr}
\begin{enumerate}
\item Der Verein führt den Namen \glqq warpzone\grqq . Der Verein wird in das Vereinsregister eingetragen und dann um den Zusatz \glqq e.V.\grqq\ ergänzt. Der Verein hat seinen Sitz in Münster.
\item Das Geschäftsjahr ist das Kalenderjahr. Das erste Geschäftsjahr beginnt mit der Eintragung des Vereins in das Vereinsregister und endet am 31.12. diesen Jahres.
\end{enumerate}

\section*{§~2~-- Zweck und Gemeinnützigkeit}
\begin{enumerate}
\item Zweck des Vereins ist die Förderung von Wissenschaft und Forschung, Kunst und Kultur,  Erziehung und Volksbildung sowie Bürgerschaftlichem Engangement und Völkerverständigung. Der Vereinszweck soll unter anderem durch folgende Mittel erreicht werden:
\begin{enumerate}
    \item Veranstaltung von Konferenzen und Bildung von Arbeitsgruppen zum Austausch von Wissen in den Bereichen: Computer Software, Computer Hardware, Netzwerk und Funktechnik, Elektrotechnik, Robotik sowie zugehöriger Fertigungs- und Entwicklungsmethoden.
    \item Organisation von Vorträgen und Diskussionen  in den  Bereichen Datenschutz, Kommunikation und aktueller Probleme der Informatik und ihre gesellschaftlichen Auswirkungen.
    \item Veranstaltung mehrtägiger Workshops zur Bearbeitung vorgegebener Aufgabenstellungen speziell auch für Jugendliche.
    \item Kunst und Kultur durch Erschaffen von zum Beipiel computergesteuerten Licht-, Medien- und Kunstinstallation.
    \item Austausch und Kontakt mit Gruppen und Vereinen ähnlicher Zielsetzung auf regionaler, nationaler und internationaler Ebene.
    \item Bürgerschaftliches Engagement durch Schaffung von Bürgernetzen, Erforschung und Weiterentwicklung derselben sowohl auf technischer als auch gesellschaftlicher Ebene sowie Wissensvermittlung und Beratung von interessierte Bürgern und Organsiationen.
\end{enumerate}
\item Der Verein verfolgt ausschließlich und unmittelbar gemeinnützige Zwecke im Sinne des Abschnitts \glqq Steuerbegünstigte Zwecke\grqq\ der Abgabenordnung. Er darf keine Gewinne erzielen; er ist selbstlos tätig und verfolgt nicht in erster Linie eigenwirtschaftliche Zwecke. Die Mittel des Vereins werden ausschließlich und unmittelbar zu den satzungsgemäßen Zwecken verwendet. Die Mitglieder erhalten keine Zuwendung aus den Mitteln des Vereins. Die Zahlung pauschaler Aufwandsentschädigungen an Mitglieder des Vorstandes oder anderweitig für den Verein tätige Mitglieder in angemessener Höhe ist zulässig. Niemand darf durch Ausgaben, die dem Zwecke des Vereins fremd sind oder durch unverhältnismäßig hohe Vergütungen begünstigt werden.
\end{enumerate}

\section*{§~3~-- Mitgliedschaft}
\begin{enumerate}
\item Ordentliche Vereinsmitglieder können natürliche und juristische Personen, Handelsgesellschaften, nicht rechtsfähige Vereine sowie Anstalten und Körperschaften des öffentlichen Rechts werden.
\item Die Beitrittserklärung erfolgt in Textform gegenüber dem Vorstand. Über die Annahme der Beitrittserklärung entscheidet der Vorstand. Die Mitgliedschaft beginnt mit der Annahme der Beitrittserklärung.
\item Die Mitgliedschaft endet durch Austrittserklärung, durch Tod von natürlichen Personen oder durch Auflösung und Erlöschen von juristischen Personen, Handelsgesellschaften, nicht rechtsfähigen Vereinen sowie Anstalten und Körperschaften des öffentlichen Rechts oder durch Ausschluss; die Beitragspflicht für das laufende Geschäftsjahr bleibt hiervon unberührt.
\item Der Austritt wird durch schriftliche Willenserklärung gegenüber dem Vorstand mit 3 monatiger Frist ab dem Letzten des Monats vollzogen. Zu viel gezahlte Jahresbeiträge werden anteilig erstattet.
\item Die Mitgliederversammlung kann solche Personen, die sich besondere Verdienste um den Verein oder um die von ihm verfolgten satzungsgemäßen Zwecke erworben haben, zu Ehrenmitgliedern ernennen. Ehrenmitglieder haben alle Rechte eines ordentlichen Mitglieds. Sie sind von Beitragsleistungen befreit.
\item Fördermitglieder sind passive Mitglieder ohne Stimmrecht in der Mitgliederversammlung. Für ihren Beitritt gilt §~3 Abs.~2
\item Die formelle Kommunikation mit den Mitgliedern soll vorwiegend per Email stattfinden. Die Mitglieder sollen dazu eine gültige und regelmäßig gelesene Email Adresse vorhalten.
\end{enumerate}

\section*{§~4~-- Rechte und Pflichten der Mitglieder}
\begin{enumerate}
\item Die Mitglieder sind berechtigt, die Leistungen des Vereins in Anspruch zu nehmen.
\item Die Mitglieder sind verpflichtet, die satzungsgemäßen Zwecke des Vereins zu unterstützen und zu fördern, so wie die festgesetzten Beiträge in Geld zu zahlen.
\end{enumerate}

\section*{§~5~-- Ausschluss eines Mitglieds}
\begin{enumerate}
\item Ein Mitglied kann durch Beschluss des Vorstandes nach Anhörung ausgeschlossen werden, wenn es das Ansehen des Vereins schädigt, seinen Beitragsverpflichtungen für sechs Monate nicht nachkommt oder wenn ein sonstiger wichtiger Grund vorliegt.
\item Gegen den Beschluss des Vorstandes ist innerhalb einer Frist von zwei Monaten nach Zugang des Ausschließungsbeschlusses die Anrufung der Mitgliederversammlung zulässig. Bis zum Beschluss der Mitgliederversammlung ruht die Mitgliedschaft. Die Mitgliederversammlung entscheidet über die Ausschließung endgültig.
\end{enumerate}

\section*{§~6~-- Beitrag}
\begin{enumerate}
\item Der Verein erhebt einen Aufnahme- und einen Jahresbeitrag in Geld. Das Nähere regelt eine Beitragsordnung, die von der Mitgliederversammlung beschlossen wird.
\item Im begründeten Einzelfall kann für ein Mitglied durch Vorstandsbeschluss ein von der Beitragsordnung abweichender Beitrag festgesetzt werden.
\end{enumerate}

\section*{§~7~-- Organe des Vereins}
Die Organe des Vereins sind:
\begin{enumerate}
    \renewcommand*{\labelenumi}{\theenumi.}
    \addtolength{\leftmargini}{-3pt}
    \addtolength{\leftmargini}{\parindent}
    \item die Mitgliederversammlung
    \item der Vorstand.
\end{enumerate}

\section*{§~8~-- Mitgliederversammlung}
\begin{enumerate}
\item Oberstes Beschlussorgan ist die Mitgliederversammlung. Ihrer Beschlussfassung unterliegen alle in dieser Satzung oder Gesetz vorgesehenen Gegenstände, insbesondere
\begin{enumerate}
    \item die Genehmigung des Finanzberichtes,
    \item die Entlastung des Vorstandes,
    \item die Wahl und die Abberufung der Vorstandsmitglieder,
    \item die Bestellung von Finanzprüfern,
    \item Satzungsänderungen,
    \item die Genehmigung der Beitragsordnung,
    \item Beschlüsse über Anträge des Vorstandes und der Mitglieder,
    \item die Ernennung von Ehrenmitgliedern,
    \item die Auflösung des Vereins und die Beschlussfassung über die eventuelle Fortsetzung des aufgelösten Vereins.
\end{enumerate}
\item Die ordentliche Mitgliederversammlung findet einmal im Jahr statt. Außerordentliche Mitgliederversammlungen werden auf Beschluss des Vorstandes abgehalten, wenn die Interessen des Vereins dies erfordern, oder wenn mindestens 10\% der Mitglieder dies unter Angabe des Zwecks und der Gründe schriftlich beantragen. Die Einberufung der Mitgliederversammlung erfolgt unter Einhaltung der in §~3 bestimmten Textform und unter Angabe der Tagesordnung durch den Vorsitzenden mit einer Frist von 2~Wochen. Anträge zur Tagesordnung sind mindestens drei Tage vor der Mitgliederversammlung beim Vorstand einzureichen. Über die Behandlung von Initiativanträgen entscheidet die Mitgliederversammlung.
\item Beschlüsse über Satzungsänderungen einschließlich des Vereinszwecks und über die Auflösung des Vereins können nur in einer Mitgliederversammlung beschlossen werden, in der diese Tagesordnungspunkte ausdrücklich angekündigt worden sind. Solche Beschlüsse bedürfen zu ihrer Rechtswirksamkeit der Dreiviertelmehrheit der anwesenden Mitglieder.
\item Vorbehaltlich Absatz~3 bedürfen die Beschlüsse einer Mitgliederversammlung der einfachen Mehrheit der Stimmen der erschienenen Mitglieder.
Für die Wahl des Vorstands und der Kassenprüfenden wird abweichend das Approval-Voting-Verfahren angewendet:
Jedes anwesende stimmberechtigte Mitglied darf beliebig viele Stimmen abgeben, für jeden Kandidaten jedoch maximal eine.
Die Wahl gewinnt der Kandidat, der von den meisten, aber mindestens 50\% der Wählenden gewählt wird.
Die Ämter des 1. und 2. Vorsitzenden und die der Kassenprüfenden können in je einem Wahlgang gemeinsam gewählt werden.
Der Kandidat mit den meisten Stimmen ist dann der 1. Vorsitzende bzw. 1. Kassenprüfer, der mit den zweitmeisten Stimmen der 2. Vorsitzende bzw. 2. Kassenprüfer.
Sollte der Kandidat mit den zweitmeisten Stimmen nicht von mindestens 50\% der Wählenden gewählt worden sein, so wird nur diese Wahl wiederholt, der erstplazierte Kandidat bleibt gewählt.
Bei Stimmengleichheit erfolgt eine Stichwahl zwischen den Kanidierenden mit den meisten Stimmen.
\item Jedes Mitglied hat eine Stimme. Juristische Personen haben einen Stimmberechtigten schriftlich zu bestellen.
\item Die Mitgliederversammlung wird von dem Vorsitzenden, bei seiner Abwesenheit von dem stellvertretenden Vorsitzenden geleitet. Der Schriftführer ist Protokollführer. Die Mitgliederversammlung kann einen anderen Versammlungsleiter oder einen anderen Protokollführer bestimmen.
\item Auf Antrag eines Mitglieds ist geheim abzustimmen. Über die Beschlüsse der Mitgliederversammlung ist ein Protokoll anzufertigen, das vom Versammlungsleiter und dem Protokollführer zu unterzeichnen ist; das Protokoll ist allen Mitgliedern zugänglich zu machen.
\end{enumerate}

\section*{§~9~-- Vorstand}
\begin{enumerate}
\item Der Vorstand besteht aus fünf Mitgliedern, und zwar:
\begin{enumerate}
    \item dem ersten Vorsitzenden,
    \item dem zweiten Vorsitzenden
    \item dem Schriftführer,
    \item dem Schatzmeister,
    \item einem Beisitzer.
\end{enumerate}
\item Vorstand im Sinne des §~26 Abs.~2 BGB ist der erste und der zweite Vorsitzende. Sie vertreten den Verein gemeinschaftlich. Die Vorstandsmitglieder sind von der Vorschrift des §~181 BGB befreit.
\item Die Amtsdauer der Vorstandsmitglieder beträgt zwei Jahre; Wiederwahl ist 
zulässig. In ungeraden Jahren werden der 1.Vorsitzende, der Schatzmeister und 
der Beisitzer gewählt. In geraden Jahren wird der 2.Vorsitzende und der 
Schriftführer gewählt. 
\item Eine vorzeitige Neuwahl eines Postens ändert den Wahlrhythmus nicht.
\item Besteht der Vorstand aus weniger als zwei Mitgliedern, so sind 
unverzüglich Nachwahlen durchzuführen.
\item Beschlüsse des Vorstands werden mit der Mehrheit der Stimmen der an der Beschlussfassung teilnehmenden Vorstandsmitglieder gefasst. Bei Stimmengleichheit gibt die Stimme des Vorsitzenden, bei seiner Abwesenheit die des zweiten Vorsitzenden den Ausschlag.
\item Der Schatzmeister überwacht die Haushaltsführung und verwaltet unter Beachtung etwaiger Vorstandsbeschlüsse das Vermögen des Vereins. Er hat auf eine sparsame und wirtschaftliche Haushaltsführung hinzuwirken. Mit Ablauf des Geschäftsjahres stellt er unverzüglich die Abrechnung sowie die Vermögensübersicht und sonstige Unterlagen von wirtschaftlichem Belang den Finanzprüfern des Vereins zur Verfügung.
\item Die Vorstandsmitglieder sind ehrenamtlich tätig.
\item Der Vorstand kann einen \glqq Wissenschaftlichen Beirat\grqq einrichten, der für den Verein beratend und unterstützend tätig wird; in den Beirat können auch Nicht-Mitglieder berufen werden.
\item Der Schatzmeister ist für den administrativen und finanziellen Bereich des Vereins verantwortlich. Die Rahmenbedingungen der von ihm oder dem Vorstand getätigten Geldtransaktionen ist von der Mitgliederversammlung zu beschließen.
\end{enumerate}

\section*{§~10~-- Finanzprüfer}
\begin{enumerate}
\item Zur Kontrolle der Haushaltsführung bestellt die Mitgliederversammlung zwei Finanzprüfer. Nach Durchführung ihrer Prüfung geben sie dem Vorstand Kenntnis von ihrem Prüfungsergebnis und erstatten der Mitgliederversammlung Bericht.
\item Die Finanzprüfer dürfen dem Vorstand nicht angehören.
\end{enumerate}

\section*{§~11~-- Auflösung des Vereins}
Bei Auflösung oder Aufhebung des Vereins oder bei Wegfall steuerbegünstigter Zwecke fällt das Vermögen der Körperschaft an den digitalcourage~e.V.\ der es unmittelbar und ausschließlich für gemeinnützige Zwecke zu verwenden hat.

\end{document}